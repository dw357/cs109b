%% LyX 2.1.4 created this file.  For more info, see http://www.lyx.org/.
%% Do not edit unless you really know what you are doing.
\documentclass[english]{revtex4-1}
\usepackage[T1]{fontenc}
\usepackage[latin9]{inputenc}
\setcounter{secnumdepth}{3}
\usepackage{graphicx}
\usepackage{babel}
\begin{document}

\title{The Case For Combating Malaria}

\maketitle
Concentrated in the poorest regions of the world, MALARIA is a leading
cause of death and ill-health in the developing world. It needs our
attention, NOW. 


\section*{Malaria situation today}

Malaria is a life-threatening disease caused by parasites that are
transmitted to people through the bites of infected female mosquitoes.
About 3.2 billion people \textendash{} almost half of the world\textquoteright s
population \textendash{} are at risk of malaria. Young children, pregnant
women and non-immune travelers from malaria-free areas are particularly
vulnerable to the disease when they become infected. 

In the year 2015, there are a total of 214 million malaria cases reported
worldwide. A look at the regional distribution of malaria cases shows
that tropical regions are most affected. Among all , Africa has significantly
more malaria cases than elsewhere, consituting about 90\% of total
global occurance. Within Africa, Nigeria has the most number of malaria
cases, more than three times that of the Democratic Republic of Congo,
the second highest country on the list. Although the number of malaria
cases in Nigeria has decreased overtime, it has always been the country
most plagued by the disease. India also has a large number of malaria
cases, the highest in Asia. 

\begin{figure}
\centering{}\includegraphics[scale=0.3]{\string"Dashboard 1\string".eps}
\end{figure}



\section*{Increased funding reduces malaria}

Malaria is preventable and curable, and increased efforts are dramatically
reducing the malaria burden in many places. 

Global funding efforts increased from 4,352 million in year 2005 to
12,894 million in year 2013. This includes demestic resources, global
fund, funding from the world bank, United States, United Kingdom,
as well as numerous other sources. Domestic funds increases steadily
from year 2005 to year 2013, constituting a significant portion of
the total monetary efforts towards fighting malaria. Global funds
increases from the year 2005 to 2009, but saw a dipping after 2009
possibly due to the global financial crisis affecrting major countires
around the world. However, since 2011, global funding efforts has
started to increase again and now contributes the most significant
part among all funding efforts. 

The increase in funding saw a significant decrease in the number of
malaria cases. The numbere of malaria cases has reduced drastically
from 263 million in 2005 to 211 million in 2015, by 20\%. The global
decrease in malaria incidence, which is the number of emerging cases
per year, has decreased by 37\% between 2000 and 2015. This is also
a 60\% decrease in global malaria mortality rate between the same
time period. Increased funding indeed helps to reduce the number of
malaria occurances.

\begin{figure}
\centering{}\includegraphics[scale=0.3]{\string"Dashboard 2\string".eps}
\end{figure}



\section*{Long way to go in fight against malaria }

Today we are still faced with 

tangible reduction in cases 
\end{document}
